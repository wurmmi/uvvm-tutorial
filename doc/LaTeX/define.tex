\documentclass[11pt,a4paper,oneside]{article}

% Global format settings
\setlength\parindent{0pt}

% Packages
\usepackage[left=2.5cm,
			right=2.5cm,
			bottom=2cm,
			top=2cm,
			includeheadfoot]{geometry}
\usepackage{booktabs}
\usepackage[ngerman]{babel}
\usepackage[utf8]{inputenc}
\usepackage{textcomp} %fuer Sonderzeichen wie bspw. µ
\usepackage{ucs}
\usepackage[usenames,dvipsnames]{xcolor}
\usepackage{amsmath}
\usepackage{amsthm}
\usepackage{amsbsy}
\usepackage{amssymb}
\usepackage{amsfonts}
\usepackage{color}
\usepackage{float}
\usepackage{listings}
\usepackage[T1]{fontenc}
\usepackage{listings}
\usepackage{graphicx}
\usepackage{caption}
\usepackage{subfigure}
%\usepackage{subcaption}
\usepackage{titlesec}
\usepackage{pdfpages}   % Zum Einbinden von pdfs
\usepackage{tikz}       % Overlay text in pdfs
\usepackage{fancyhdr}   % header, footer
\usepackage{xargs}		% for more than one optional argument in \newcommand

\usepackage[pdfborder={0 0 0 [0 1]},
			pdfauthor={Wurm},
			pdftitle={RRT5 Uebung},
			pdfsubject={RRT5 Uebung},
			colorlinks=true,
			linkcolor=black,
			citecolor=black,
			urlcolor=blue] {hyperref}

\definecolor{mygray}{rgb}{0.97,0.97,0.97}

% settings for sectioning
\setcounter{secnumdepth}{4} % how many sectioning levels to assign numbers to
\setcounter{tocdepth}{4}    % how many sectioning levels to show in TableOfContent
\titlespacing*{\paragraph}{0pt}{2\baselineskip}{3\baselineskip}

%new command: extracts Filename from Path
\makeatletter
\DeclareRobustCommand{\getFileNameFromPath}{%
	\begingroup
	% \lstname seems to change hyphens into \textendash
	\def\textendash{-}%
	\filename@parse{\lstname}%
	\texttt{\filename@base.\filename@ext}%
	\endgroup
}
\makeatother

\lstset
{
	literate={ö}{{\"o}}1
	{ä}{{\"a}}1
	{ü}{{\"u}}1
	{ß}{{\ss}}1
	{é}{{\'e}}1
	{€}{{\euro}}1,
	inputencoding=ansinew,
	extendedchars=true,
	basicstyle=\scriptsize\ttfamily,
	numberstyle=\scriptsize,
	breaklines=true,
	tabsize=2,
	numbersep=5pt
}

\lstdefinestyle{customcpp}
{
	language=C++,					          % language
	backgroundcolor=\color{mygray},			  % background color of text
	frame=single,							  % surrounding of text
	xleftmargin=0.6cm,
	numbers=left,							  % show line numbers
	firstnumber=1,							  % starting number
	stepnumber=5,							  % starting number + n*stepnumber
	stringstyle=\color{BrickRed}\ttfamily,	  % style of strings
	commentstyle=\color{OliveGreen}\ttfamily, % style of comments
	sensitive=true,							  % true is keyword, TruE is not
	keywordstyle=\color{blue}\bfseries,		  % style of keywords
	otherkeywords={	bool,					  % define additional keywords
					NULL,
					using,
					public,
					private,
					class,
					namespace,
					nullptr,
					size_t,
					this,
					true,
					false,
				    \#ifdef,
				    \#ifndef,
				    \#endif },
	tabsize=4,								  % one tab equals 4 spaces
	showspaces=false,						  % show symbol ,_, instead of space
	showstringspaces=false,					  % show symbol ,_, instead of space in strings
	showtabs=false,							  % show tabs as line
	breaklines=true,						  % automatic line breaking of too long lines
	captionpos=b,							  % position of title b=bottom, t=top
	title=\getFileNameFromPath                % title
}

\lstdefinestyle{custommatlab}
{
	language=Matlab,
	frame=single,							  % surrounding of text
	backgroundcolor=\color{mygray},			  % background color of text
	xleftmargin=0.6cm,
	numbers=left,							  % show line numbers
	firstnumber=1,							  % starting number
	stepnumber=5,							  % starting number + n*stepnumber
	breaklines=true,%
	morekeywords={matlab2tikz},
	keywordstyle=\color{blue},%
	morekeywords=[2]{1}, keywordstyle=[2]{\color{black}},
	identifierstyle=\color{black},%
	stringstyle=\color{BrickRed},
	commentstyle=\color{OliveGreen},%
	emph=[1]{for,end,break},emphstyle=[1]\color{red}, %some words to emphasise
	tabsize=4,								  % one tab equals 4 spaces
	showspaces=false,						  % show symbol ,_, instead of space
	showstringspaces=false,					  % show symbol ,_, instead of space in strings
	showtabs=false,							  % show tabs as line
	breaklines=true,						  % automatic line breaking of too long lines
	captionpos=b,							  % position of title b=bottom, t=top
	title=\getFileNameFromPath                % title
}

\lstdefinestyle{custombash}
{
	language=bash,
	frame=single,							  % surrounding of text
	backgroundcolor=\color{mygray},			  % background color of text
	xleftmargin=0.6cm,
	numbers=left,							  % show line numbers
	firstnumber=1,							  % starting number
	stepnumber=5,							  % starting number + n*stepnumber
	breaklines=true,%
	morekeywords={matlab2tikz},
	keywordstyle=\color{blue},%
	morekeywords=[2]{1}, keywordstyle=[2]{\color{black}},
	identifierstyle=\color{black},%
	stringstyle=\color{BrickRed},
	commentstyle=\color{OliveGreen},%
	emph=[1]{for,end,break},emphstyle=[1]\color{red}, %some words to emphasise
	tabsize=4,								  % one tab equals 4 spaces
	showspaces=false,						  % show symbol ,_, instead of space
	showstringspaces=false,					  % show symbol ,_, instead of space in strings
	showtabs=false,							  % show tabs as line
	breaklines=true,						  % automatic line breaking of too long lines
	captionpos=b,							  % position of title b=bottom, t=top
	title=\getFileNameFromPath                % title
}

\lstdefinestyle{customoutput}
{
	backgroundcolor=\color{mygray},
	frame=single,
	numbers=left,
	firstnumber=1,
	stepnumber=5,
	captionpos=b,
	title=\getFileNameFromPath
}

% usage: \sourceCodeXXX[1-20, 40-50]{path}
\newcommandx{\sourceCodeCpp}[3][1=1, 2=1]{\lstinputlisting[style=customcpp,linerange={#1}, firstnumber=#2] {#3}}
\newcommandx{\sourceCodeMatlab}[3][1=1, 2=1]{\lstinputlisting[style=custommatlab,linerange={#1}, firstnumber=#2] {#3}}
\newcommandx{\sourceCodeBash}[3][1=1, 2=1]{\lstinputlisting[style=custombash,linerange={#1}, firstnumber=#2] {#3}}

% usage: \textFile[2cm]{path}
\newcommand{\textFile}[2][0cm]{\lstinputlisting[xleftmargin=#1, xrightmargin=#1, style=customoutput]{#2}}

% usage: \horizontalLine{0.5}
\newcommand{\horizontalLine}[1]{\rule{\linewidth}{#1}}

% usage: \includepicture [width=1] [angle=0] {caption text} {filepath}
\newcommandx{\includepicture}[4][1=1,2=0]
{
	\begin{figure}[H]
		\centering
		\includegraphics[width=#1\textwidth,angle=#2]{\string"#4"}
		\caption{\textit{#3}}
	%	\label{key}
	\end{figure}
}

% usage: \includepictureLR [angleL=0][angleR=0] {captionL text}{captionR text}{caption middle} {filepathL}{filepathR}
\newcommandx{\includepictureLR}[7][1=0,2=0]
{
	\begin{figure}[H]
		\subfigure[#3]{\includegraphics[width=0.5\textwidth,angle=#1]{\string"#6"}}
		\subfigure[#4]{\includegraphics[width=0.5\textwidth,angle=#2]{\string"#7"}}
		\caption{#5}
	\end{figure}
}

% command for 4th level section -->  1.2.3.4 SectionsName
% usage:  \subsubsubsection[0 cm]{path} % if text follows
%         \subsubsubsection{path}       % else
\newcommand{\subsubsubsection}[2][0.3cm]{\paragraph{#2}\mbox{} \vspace{#1} \newline }
