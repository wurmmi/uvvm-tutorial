\section{Tutorial}

This tutorial guides you through the example project that was developed.

\subsection{Device Under Verification (DUV)}

describes the DUV (block diagram)

\subsection{Testbench Architecture}

describes the testbench architecture as a block diagram

\subsection{Hands-On}

At this point, we have enough information and are ready to start working with the example project.

\subsubsection{Project environment}

Let's start with getting to know the projects' environment setup. \\
In this part we will compile the UVVM library and the DUV, and run the simulation of our first example testbench!

\begin{enumerate}
  \item Open the example-tb folder in the repository.
        You will find a Makefile, that's used to simplify the usage of this project. It builds an abstraction layer to the underlaying ModelSim commands.
  \item \texttt{make help} is your friend. \\
        It shows all the actions, that this project provides.
  \item Call \texttt{make ip}. \\
        This compiles all the VHDL files, that belong to the DUV.\\
        Call this target, whenever your DUVs RTL code changed and you want to simulate it.
  \item Call \texttt{make uvvm}. \\
        This compiles several required components of the UVVM library. \\
        That only needs to be done once.
  \item Call \texttt{make test}. \\
        This compiles the testbench files and runs the simulation.
  \item Read through the log that was generated.\\
        The console output was also logged into the file \textit{example-tb/build/log/sim\_testbench.log}.
\end{enumerate}

