\section{Prerequisites}

This tutorial requires the ModelSim simulator software to be installed.
Furthermore, it is recommended to work on a PC that runs a Linux operating system.

\section{Preparation}

This tutorial aims to explain UVVM and work with it.
Thus, we do not want to lose time with the environment setup.
So in order to be able to follow the tutorial, please complete the
following steps in advance.

\subsection{Install ModelSim}

Any version of ModelSim should satisfy this tutorials' requirement.\\

In case you are using ModelSim Intel Starter Version, you might need to
create a symlink in the installation directory.
\begin{itemize}
  \item Go to the ModelSim installation directory\\
        (e.g. \texttt{/opt/intelFPGA/19.1/modelsim\_ase})
  \item Create a symlink from \texttt{linuxaloem} to \texttt{linuxpe}.\\
        (\texttt{sudo ln -s linuxaloem/ linuxpe})
\end{itemize}

Verify, that ModelSim can be found. Execute the following command in a command line.\\
\\
\texttt{vsim -version}

\subsection{Clone the tutorials' repository}

Several sources required by this tutorial can be found in a GitHub repository.\\
Download, or clone the repository.\\
\\
\texttt{git clone \textendash \textendash recursive https://github.com/wurmmi/uvvm-tutorial.git }
